\documentclass[sigconf]{acmart}

\title{Predictive Maintenance for Hydraulic Systems Using Machine Learning}

\author{Rock Deng}
\affiliation{%
  \institution{University of Colorado Boulder}}
\email{dengyongpeng110@outlook.com}

\begin{document}

\begin{abstract}
Hydraulic systems are critical components in industrial applications, and unexpected failures can lead to costly downtime. This project explores the use of machine learning for predictive maintenance, leveraging the Condition Monitoring of Hydraulic Systems dataset. The objective is to classify failure risks and suggest preventive measures. By implementing various machine learning techniques, including decision trees, XGBoost, and time-series models, this study aims to enhance failure prediction accuracy while maintaining computational efficiency.
\end{abstract}

\maketitle

\section{Introduction}
Hydraulic systems play a vital role in industrial operations, but their maintenance often relies on reactive or scheduled maintenance, leading to inefficiencies. Predictive maintenance, powered by machine learning, offers an intelligent approach to anticipating failures before they occur. This study aims to implement and evaluate machine learning techniques to develop a predictive maintenance model using the Condition Monitoring of Hydraulic Systems dataset.

\section{Related Work}
Several studies have explored predictive maintenance using machine learning:
\begin{itemize}
    \item \textbf{Machine Learning Models}: Random Forests, Support Vector Machines (SVMs), and Neural Networks have been applied to classify failures \cite{helwig2015automated}.
    \item \textbf{Feature Engineering}: Feature selection techniques such as Principal Component Analysis (PCA) have been used to improve model efficiency \cite{alenany2021comprehensive}.
    \item \textbf{Time-Series Forecasting}: Long Short-Term Memory (LSTM) networks and ARIMA models have been tested for early failure detection \cite{goodarzi2023deep}.
\end{itemize}
Our work builds on these studies by exploring lightweight and interpretable models suitable for real-time deployment.

\section{Proposed Work}
\subsection{Datasets and Tools}
\begin{itemize}
    \item \textbf{Dataset}: Condition Monitoring of Hydraulic Systems.
    \item \textbf{Tools}: Python, Pandas, NumPy, Scikit-learn, TensorFlow, PyTorch.
    \item \textbf{Time-Series Analysis Libraries}: statsmodels, Prophet.
\end{itemize}

\subsection{Main Tasks}
\begin{itemize}
    \item Data preprocessing: Handle missing values, normalize sensor readings, perform feature engineering.
    \item Model selection: Compare Decision Trees, XGBoost, ARIMA, and LSTM models.
    \item Performance evaluation: Use accuracy, precision-recall curves, RMSE.
    \item Deployment considerations: Optimize for real-time monitoring.
\end{itemize}

\section{Evaluation}
\begin{itemize}
    \item \textbf{Classification Metrics}: Accuracy, Precision, Recall, F1-score.
    \item \textbf{Time-Series Metrics}: Root Mean Squared Error (RMSE).
    \item \textbf{Model Efficiency}: Training time and inference speed.
\end{itemize}
Comparison of different machine learning approaches will be conducted to determine the most effective and efficient model.

\section{Discussion}
\subsection{Project Timeline}
\begin{table}[h]
    \centering
    \begin{tabular}{|c|l|}
    \hline
    Week & Task \\
    \hline
    1 & Dataset exploration and preprocessing \\
    2-3 & Implement baseline models (Decision Trees, XGBoost) \\
    4 & Experiment with time-series models (LSTM, ARIMA) \\
    5 & Evaluation and performance comparison \\
    6 & Final report writing and refinements \\
    \hline
    \end{tabular}
    \caption{Project Timeline}
\end{table}

\subsection{Potential Challenges and Mitigations}
\begin{itemize}
    \item \textbf{Data Imbalance}: Rare failure cases may affect model performance.
    \begin{itemize}
        \item Solution: Apply oversampling techniques such as SMOTE.
    \end{itemize}
    \item \textbf{Computational Constraints}: Deep learning models may be resource-intensive.
    \begin{itemize}
        \item Solution: Prioritize lightweight models with feature selection.
    \end{itemize}
\end{itemize}

\subsection{Alternative Approaches}
If supervised learning models fail, anomaly detection methods such as Autoencoders and Isolation Forest will be explored.

\section{Conclusion}
This project aims to develop a predictive maintenance system for hydraulic systems using machine learning. By leveraging sensor data, the system will predict failures and suggest preventive measures. Future work includes real-time model deployment and IoT integration for continuous monitoring.

\bibliographystyle{ACM-Reference-Format}
\begin{thebibliography}{9}
    \bibitem{helwig2015automated} Helwig, N. E., et al. (2015). Automated training of condition monitoring systems using multivariate statistics. \textit{IEEE Transactions on Industrial Informatics}.
    \bibitem{alenany2021comprehensive} Alenany, A., et al. (2021). Comprehensive analysis for sensor-based hydraulic system condition monitoring. \textit{ResearchGate}.
    \bibitem{goodarzi2023deep} Goodarzi, H., et al. (2023). Deep convolutional networks for cyclic sensor data. \textit{arXiv preprint arXiv:2308.06987}.
\end{thebibliography}

\end{document}

